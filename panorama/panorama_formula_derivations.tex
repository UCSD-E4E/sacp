\documentclass{article}
\usepackage{amsmath}
\usepackage{amssymb}



\title{Equations for Spherical Panoramas}
\author{Sergei Radutnuy}

\begin{document}
\maketitle
\section{Lens Basics/introduction - ?}
what a lens does to an image. this is probably gonna be a subset of projections
\section{Field of View}
what it means intuitively, rigorously.the field of view 
equation how it's derived
\section{Projections onto the Sphere}
draw a few spheres illustrating why it's not just pitch += constant 
and yaw += constant; lines of latitude vs. lines of longitude \& rectangular
sectors.
\\
easiest way to explain this, just map the 4 corners of the image...
\section{Determining Coordinates for a Panorama}
WLOG assume you're incrementing the pitch by the same amount each time
(what the code actually does) and then determining how much to increment the
pitch for every row. use the 4-corner / top/bottom mapping examples. how 
overlap was determined naively

\section{expansion - how to give area of overlap? does anyone care?}
some nasty looking solid angle integrals
\section{another expansion - what if you have a weird lens}
\end{document}